% !TEX program = pdflatex
\documentclass[12pt,a4paper,oneside]{report}

% ============================================================
%  SMART TRAVEL PLATFORM – RAG / PLAN-RAG
%  Full report template (Vietnamese) – Optimized for pdfLaTeX
%  Compile: pdfLaTeX -> BibLaTeX (biber) -> pdfLaTeX x2
% ============================================================

% -------------------- Page & typography ----------------------
\usepackage[a4paper,margin=2.5cm]{geometry}
\usepackage{setspace}
\onehalfspacing
\usepackage{microtype}

% -------------------- Vietnamese (pdfLaTeX) ------------------
\usepackage[utf8]{inputenc}
\usepackage[T5]{fontenc}
\usepackage[vietnamese]{babel}

% -------------------- Figures, tables ------------------------
\usepackage{graphicx}
\usepackage{float}
\usepackage{caption}
\usepackage{subcaption}
\usepackage{booktabs}
\usepackage{longtable}
\usepackage{tabularx}
\usepackage{multirow}

% -------------------- Math & symbols -------------------------
\usepackage{amsmath,amssymb}

% -------------------- Diagrams (pipeline placeholders) -------
\usepackage{tikz}
\usetikzlibrary{positioning,arrows.meta,shapes.geometric}

% -------------------- Code listings --------------------------
\usepackage{listings}
\usepackage{xcolor}
\lstset{
  basicstyle=\ttfamily\small,
  breaklines=true,
  frame=single,
  rulecolor=\color{black!20},
  columns=fullflexible,
  literate=%
    {á}{{\'a}}1 {à}{{\`a}}1 {ả}{{\h{a}}}1 {ã}{{\~a}}1 {ạ}{{\d{a}}}1
    {ă}{{\u{a}}}1 {ắ}{{\'{\\u{a}}}}1 {ằ}{{\`{\\u{a}}}}1 {ẳ}{{\h{\\u{a}}}}1 {ẵ}{{\~{\\u{a}}}}1 {ặ}{{\d{\\u{a}}}}1
    {â}{{\^a}}1 {ấ}{{\'{\\^a}}}1 {ầ}{{\`{\\^a}}}1 {ẩ}{{\h{\\^a}}}1 {ẫ}{{\~{\\^a}}}1 {ậ}{{\d{\\^a}}}1
    {đ}{{\dj}}1
    {é}{{\'e}}1 {è}{{\`e}}1 {ẻ}{{\h{e}}}1 {ẽ}{{\~e}}1 {ẹ}{{\d{e}}}1
    {ê}{{\^e}}1 {ế}{{\'{\\^e}}}1 {ề}{{\`{\\^e}}}1 {ể}{{\h{\\^e}}}1 {ễ}{{\~{\\^e}}}1 {ệ}{{\d{\\^e}}}1
    {í}{{\'i}}1 {ì}{{\`i}}1 {ỉ}{{\h{i}}}1 {ĩ}{{\~i}}1 {ị}{{\d{i}}}1
    {ó}{{\'o}}1 {ò}{{\`o}}1 {ỏ}{{\h{o}}}1 {õ}{{\~o}}1 {ọ}{{\d{o}}}1
    {ô}{{\^o}}1 {ố}{{\'{\\^o}}}1 {ồ}{{\`{\\^o}}}1 {ổ}{{\h{\\^o}}}1 {ỗ}{{\~{\\^o}}}1 {ộ}{{\d{\\^o}}}1
    {ơ}{{\horn{o}}}1 {ớ}{{\'{\\horn{o}}}}1 {ờ}{{\`{\\horn{o}}}}1 {ở}{{\h{\\horn{o}}}}1 {ỡ}{{\~{\\horn{o}}}}1 {ợ}{{\d{\\horn{o}}}}1
    {ú}{{\'u}}1 {ù}{{\`u}}1 {ủ}{{\h{u}}}1 {ũ}{{\~u}}1 {ụ}{{\d{u}}}1
    {ư}{{\horn{u}}}1 {ứ}{{\'{\\horn{u}}}}1 {ừ}{{\`{\\horn{u}}}}1 {ử}{{\h{\\horn{u}}}}1 {ữ}{{\~{\\horn{u}}}}1 {ự}{{\d{\\horn{u}}}}1
    {ý}{{\'y}}1 {ỳ}{{\`y}}1 {ỷ}{{\h{y}}}1 {ỹ}{{\~y}}1 {ỵ}{{\d{y}}}1
}

% -------------------- Links & references ---------------------
\usepackage[hidelinks]{hyperref}
\usepackage[nameinlink,noabbrev]{cleveref}

% -------------------- Header / footer ------------------------
\usepackage{fancyhdr}
\pagestyle{fancy}
\fancyhf{}
\chead{\leftmark}
\cfoot{\thepage}

% -------------------- Lists & utilities ----------------------
\usepackage{enumitem}
\setlist{nosep}

% -------------------- TODO notes for placeholders ------------
% Comment out todonotes if not available or causing issues
% \usepackage{todonotes}
% Alternative: simple command
\newcommand{\todo}[2][]{\textcolor{red}{\textbf{TODO: #2}}}

% -------------------- Bibliography ---------------------------
% Create a dummy references.bib file or comment out if not using
\usepackage[backend=biber,style=ieee,sorting=none]{biblatex}
% Comment the line below if references.bib doesn't exist
% \addbibresource{references.bib}

% -------------------- Custom metadata ------------------------
\newcommand{\ProjectName}{Smart Travel Platform – AI-Powered Travel Planning System}
\newcommand{\ReportDateA}{30/12/2025}
\newcommand{\ReportDateB}{12/01/2026}
\newcommand{\University}{\textit{(Điền tên trường / khoa)}}
\newcommand{\StudentName}{\textit{(Điền họ tên)}}
\newcommand{\StudentID}{\textit{(Điền MSSV)}}
\newcommand{\Advisor}{\textit{(Điền GVHD)}}

\title{Report\_12\_01\_Tour\_with\_NLP}
\author{Chơn Nhân Nguyễn}
\date{January 2026}

\begin{document}

\maketitle

% ========================== Front matter ======================
\pagenumbering{roman}

\begin{abstract}
Báo cáo trình bày hệ thống tư vấn du lịch tiếng Việt dựa trên kiến trúc Retrieval-Augmented Generation (RAG) và biến thể Plan-RAG nhằm giảm hallucination, tăng tính grounded theo dữ liệu nội bộ, và hỗ trợ truy vấn đa ý định (multi-intent) trong ngữ cảnh du lịch. 
Hệ thống gồm hai nhánh: (i) \textbf{Traveler Assistant (B2C)} để gợi ý địa điểm, khách sạn, ẩm thực, lập lịch trình và ước tính chi phí; (ii) \textbf{Business Insights (B2B)} để phân tích review khách sạn, trích pain points/KPI và đề xuất hành động.
Báo cáo cũng phân tích một lỗi quan trọng (12/01/2026) liên quan đến truy xuất thông tin chi tiết cho địa điểm/khách sạn đã chọn, kèm root cause analysis và roadmap khắc phục.
\end{abstract}

\tableofcontents
\listoffigures
\listoftables

\chapter*{Danh sách từ viết tắt}
\addcontentsline{toc}{chapter}{Danh sách từ viết tắt}
\begin{longtable}{p{0.22\textwidth}p{0.72\textwidth}}
\toprule
\textbf{Từ viết tắt} & \textbf{Giải thích} \\
\midrule
LLM & Large Language Model \\
RAG & Retrieval-Augmented Generation \\
Plan-RAG & Multi-stage RAG có bước lập kế hoạch (Planner $\rightarrow$ Experts) \\
NLU & Natural Language Understanding (trích xuất intent/entity) \\
NER & Named Entity Recognition \\
UI & User Interface (cards/options/itinerary/...) \\
B2C & Business to Consumer (Traveler) \\
B2B & Business to Business (Insights) \\
\bottomrule
\end{longtable}

\clearpage
\pagenumbering{arabic}

% ============================ Chapter 1 =======================
\chapter{Giới thiệu}

\section{Bối cảnh và vấn đề}
Thông tin du lịch phân mảnh, nhiễu và thay đổi theo thời gian; LLM-only dễ hallucination, thiếu cập nhật và khó kiểm chứng. Do đó cần cơ chế kết nối LLM với tri thức có cấu trúc và văn bản nội bộ thông qua RAG/Plan-RAG.

\section{Khoảng trống ứng dụng}
Các hệ thống phổ biến thường rơi vào hai dạng: (i) LLM-only; (ii) one-stage RAG (retrieve một lần rồi generate). Với truy vấn du lịch thực tế mang tính \textbf{multi-intent} và \textbf{multi-constraint}, one-stage RAG thường không đủ vì không tách được mục tiêu/điều kiện và thiếu kiểm soát từng sub-task.

\section{Mục tiêu và phạm vi}
\begin{itemize}
  \item (M1) Trợ lý tư vấn du lịch tiếng Việt trả lời \textbf{grounded} theo dữ liệu thật.
  \item (M2) Nâng cấp sang Plan-RAG để xử lý truy vấn phức tạp (multi-intent, multi-constraint).
  \item (M3) Hỗ trợ cả B2C (Traveler) và B2B (Business Insights).
  \item (M4) Đánh giá bằng các độ đo: Intent Accuracy, Relevance, Latency, Success Rate, Groundedness.
\end{itemize}

\section{Đóng góp chính}
\begin{itemize}
  \item Thiết kế kiến trúc Plan-RAG (NLU $\rightarrow$ Planner $\rightarrow$ Experts $\rightarrow$ Aggregator $\rightarrow$ Generator) cho domain du lịch tiếng Việt.
  \item Xây dựng hệ Expert Executors: Spot/Hotel/Food/Itinerary/Cost và cơ chế context multi-turn.
  \item Nâng cấp hỗ trợ multi-intent (MultiIntentExtractor, MultiPlannerAgent, ResponseAggregator).
  \item Benchmark so sánh RAG cũ vs Plan-RAG.
  \item Phân tích và đề xuất khắc phục lỗi truy xuất chi tiết spot/hotel đã chọn (RCA).
\end{itemize}

% ============================ Chapter 2 =======================
\chapter{Cơ sở lý thuyết và hệ thống liên quan}

\section{LLM-only}
LLM-only có ưu điểm hội thoại linh hoạt nhưng nhược điểm là dễ hallucination, khó cập nhật theo dữ liệu nội bộ, và khó kiểm chứng nguồn.

\section{RAG (Retrieval-Augmented Generation)}
Mô hình tổng quát:
\[
\text{Query} \rightarrow \text{Retrieve} \rightarrow \text{Augment Prompt} \rightarrow \text{Generate}
\]
Trong domain du lịch, retrieval có thể là keyword search, vector search, metadata filter, hoặc hybrid.

\section{One-stage vs Multi-stage / Plan-RAG}
\subsection{One-stage RAG}
\[
\text{Query} \rightarrow \text{Retrieve} \rightarrow \text{Generate}
\]
Phù hợp câu hỏi đơn giản, 1 mục tiêu. 

\subsection{Plan-RAG (Multi-stage)}
\[
\text{Query} \rightarrow \text{Plan} \rightarrow \text{Retrieve per sub-task} \rightarrow \text{Aggregate} \rightarrow \text{Generate}
\]
Ưu điểm: xử lý tốt multi-intent/multi-constraint, kiểm soát chất lượng theo từng sub-task; nhược điểm: latency cao hơn.

\section{So sánh các hướng}
\begin{table}[H]
\centering
\begin{tabular}{lccc}
\toprule
\textbf{Tiêu chí} & \textbf{LLM-only} & \textbf{RAG one-stage} & \textbf{Plan-RAG} \\
\midrule
Grounded theo DB & Thấp & Trung bình & Cao \\
Hallucination & Cao & Giảm & Giảm mạnh \\
Multi-intent & Yếu & TB & Tốt \\
Latency & Thấp & TB & Cao hơn \\
Mở rộng nghiệp vụ & TB & TB & Cao \\
\bottomrule
\end{tabular}
\caption{So sánh LLM-only, RAG one-stage và Plan-RAG.}
\end{table}

% ============================ Chapter 3 =======================
\chapter{Kiến trúc hệ thống và thiết kế}

\section{Tổng quan kiến trúc}
Hệ thống gồm 3 lớp: Frontend (Next.js), Backend dịch vụ du lịch (FastAPI Plan-RAG microservice), Backend analytics (B2B).

\begin{figure}[H]
\centering
\fbox{\parbox{0.9\textwidth}{
\textbf{Placeholder hình kiến trúc tổng quan} \\
\textcolor{red}{TODO: Chèn hình kiến trúc tổng quan (PNG/SVG) hoặc vẽ bằng TikZ.} \\
Gợi ý: User $\rightarrow$ Next.js API Routes $\rightarrow$ FastAPI Plan-RAG $\rightarrow$ MongoDB/VectorStore $\rightarrow$ Response (ui\_type + ui\_data)
}}
\caption{Kiến trúc tổng quan hệ thống (placeholder).}
\end{figure}

\section{Tech stack và cấu hình}
\subsection{Frontend}
Next.js 16 (App Router), TypeScript.

\subsection{Backend Traveler (Plan-RAG microservice)}
Python 3.11+, FastAPI + Uvicorn, Pydantic Settings.

\subsection{Backend Business (B2B)}
FastAPI, JWT auth.

\subsection{Cơ sở dữ liệu}
MongoDB Atlas (\texttt{spots\_db}) gồm \texttt{spots\_detailed}, \texttt{hotels}, \texttt{provinces\_info}. Tuỳ chọn ChromaDB / MongoDB Vector Search.

\subsection{LLM và Embedding}
LLM: FPT AI (SaoLa3.1-medium). Embedding: \texttt{keepitreal/vietnamese-sbert} (768-dim, CPU).

\subsection{Biến môi trường quan trọng}
\begin{itemize}
  \item \texttt{SPOTS\_MONGODB\_URI}, \texttt{SPOTS\_DB\_NAME}
  \item \texttt{FPT\_API\_KEY}, \texttt{FPT\_BASE\_URL}, \texttt{FPT\_MODEL\_NAME}, \texttt{FPT\_TEMPERATURE}, \texttt{FPT\_MAX\_TOKENS}
  \item \texttt{EMBEDDING\_MODEL}, \texttt{EMBEDDING\_DEVICE}
  \item \texttt{CHROMA\_PERSIST\_DIR}, \texttt{CHROMA\_HOST}, \texttt{CHROMA\_PORT} (tuỳ chọn)
\end{itemize}

\section{Knowledge Base}
\begin{table}[H]
\centering
\begin{tabular}{lrr}
\toprule
\textbf{Collection} & \textbf{Số lượng} & \textbf{Mục đích} \\
\midrule
\texttt{spots\_detailed} & 1{,}432 & địa điểm du lịch \\
\texttt{hotels} & 4{,}469 & khách sạn \\
\texttt{provinces\_info} & 63 & tỉnh/thành \\
\bottomrule
\end{tabular}
\caption{Kho dữ liệu chính trong MongoDB Atlas.}
\end{table}

\section{Plan-RAG pipeline (Traveler Assistant)}
\subsection{Thành phần}
\begin{itemize}
  \item IntentExtractor: trích xuất JSON intent (LLM + fallback regex).
  \item PlannerAgent: phân rã truy vấn thành SubTasks (DAG).
  \item Expert Executors: Spot/Hotel/Food/Itinerary/Cost.
  \item MasterController: orchestrator điều phối.
\end{itemize}

\subsection{Pipeline xử lý}
\begin{figure}[H]
\centering
\begin{tikzpicture}[
  node distance=10mm,
  box/.style={draw,rounded corners,align=center,minimum width=28mm,minimum height=9mm},
  arrow/.style={-Latex,thick}
]
\node[box] (u) {User Query};
\node[box,below=of u] (nlu) {Intent\\Extractor};
\node[box,below=of nlu] (plan) {Planner\\Agent};
\node[box,below=of plan] (exp) {Experts\\(Spot/Hotel/\\Food/Itinerary/Cost)};
\node[box,below=of exp] (agg) {Aggregate\\Context};
\node[box,below=of agg] (gen) {Generator\\(+ UI data)};
\draw[arrow] (u) -- (nlu);
\draw[arrow] (nlu) -- (plan);
\draw[arrow] (plan) -- (exp);
\draw[arrow] (exp) -- (agg);
\draw[arrow] (agg) -- (gen);
\end{tikzpicture}
\caption{Pipeline Plan-RAG (có thể thay bằng sơ đồ chi tiết hơn).}
\end{figure}

\subsection{Conversation context (multi-turn)}
Hệ thống duy trì context gồm: \texttt{destination}, \texttt{duration}, \texttt{budget}, \texttt{people\_count}, \texttt{interests}, \texttt{selected\_hotel}, \texttt{itinerary}, ... 
Ở mỗi lượt chat: nạp context từ request, cập nhật theo intent mới, và trả context mới cho frontend.

\section{Xử lý multi-intent (nâng cấp)}
\subsection{MultiIntentExtractor}
Phát hiện nhiều intent trong một query bằng pattern keyword + conjunction splitting (\textit{và}, \textit{+}, \textit{cùng}, ...).

\subsection{MultiPlannerAgent}
Tạo ExecutionPlan cho nhiều intents; các tasks cùng priority có thể chạy song song; tasks phụ thuộc được sắp xếp theo topological sort.

\subsection{ResponseAggregator}
Tổng hợp nhiều kết quả thành response có cấu trúc theo từng section (Hotels/Spots/Food/Itinerary/Cost).

\begin{figure}[H]
\centering
\fbox{\parbox{0.9\textwidth}{
\textbf{Placeholder pipeline multi-intent} \\
\textcolor{red}{TODO: Chèn sơ đồ multi-intent: Query $\rightarrow$ MultiIntentExtractor $\rightarrow$ MultiPlannerAgent $\rightarrow$ Experts (parallel/sequential) $\rightarrow$ ResponseAggregator.}
}}
\caption{Pipeline multi-intent (placeholder).}
\end{figure}

% ============================ Chapter 4 =======================
\chapter{Triển khai (Implementation) và API contract}

\section{API chính cho Plan-RAG}
Endpoint chính:
\begin{itemize}
  \item \textbf{POST} \texttt{/api/v1/chat/plan-rag}
\end{itemize}

\subsection{Request/Response (tóm tắt)}
\begin{lstlisting}[language=json,caption={Request mẫu cho /api/v1/chat/plan-rag}]
{
  "messages": [
    {"role": "user", "content": "Lich trinh 3 ngay Da Lat cho 2 nguoi"}
  ],
  "context": {
    "destination": "Da Lat",
    "duration": 3,
    "budget": 5000000,
    "people_count": 2
  }
}
\end{lstlisting}

\begin{lstlisting}[language=json,caption={Response mẫu (rút gọn)}]
{
  "reply": "Lich trinh 3 ngay tai Da Lat ...",
  "ui_type": "itinerary",
  "ui_data": {"itinerary": [], "spots": [], "hotels": []},
  "intent": "plan_trip",
  "context": {"destination": "Da Lat", "duration": 3, "budget": 5000000},
  "execution_time_ms": 3500
}
\end{lstlisting}

\section{Legacy endpoints}
\begin{itemize}
  \item \textbf{POST} \texttt{/api/v1/chat} (simple RAG cũ)
  \item \textbf{POST} \texttt{/chat} (tương thích frontend cũ; redirect về Plan-RAG controller)
\end{itemize}

\section{Data endpoints (featured content)}
\begin{itemize}
  \item \textbf{GET} \texttt{/api/provinces/featured?limit=N}
  \item \textbf{GET} \texttt{/api/spots/featured?limit=N}
  \item \textbf{GET} \texttt{/api/hotels/featured?limit=N}
\end{itemize}

\section{B2B Insights endpoints}
\begin{itemize}
  \item \textbf{POST} \texttt{/api/b2b/auth/login}
  \item \textbf{GET} \texttt{/api/b2b/insights/suggestions}
  \item \textbf{POST} \texttt{/api/b2b/insights/analyze}
  \item \textbf{GET} \texttt{/api/b2b/insights/history?limit=\ldots} (coming soon)
\end{itemize}

\section{Gợi ý chuẩn hoá \texttt{ui\_type} và \texttt{ui\_data}}
\textcolor{red}{TODO: Nếu các \texttt{ui\_type} thực tế đang khác giữa backend/FE (ví dụ: itinerary\_builder, location\_intro), nên lập bảng mapping và chuẩn hoá contract.}

% ============================ Chapter 5 =======================
\chapter{Thực nghiệm và đánh giá}

\section{Dataset}
\subsection{Structured data (MongoDB)}
Spots: 1{,}432; Hotels: 4{,}469; Provinces: 63.

\subsection{Unstructured data (Crawler)}
Module crawler thu thập bài viết theo từ khoá để tăng độ phủ về kinh nghiệm/mẹo.

\section{Độ đo}
\begin{itemize}
  \item Intent Accuracy (\%)
  \item Relevance Score (\%) (proxy)
  \item Latency (ms)
  \item Success Rate (\%)
  \item Groundedness / DB-hit rate (\%)
  \item Coverage (\%) cho truy vấn multi-intent
\end{itemize}

\section{Kết quả benchmark (từ báo cáo 30/12)}
\begin{table}[H]
\centering
\begin{tabular}{lrrr}
\toprule
\textbf{Metric} & \textbf{Old RAG} & \textbf{Plan-RAG} & \textbf{Chênh lệch} \\
\midrule
Intent Accuracy & 0.0\% & 100.0\% & +100\% \\
Avg Relevance & 9.2\% & 33.2\% & +24.0\% \\
Avg Latency & 1660 ms & 3665 ms & +2005 ms \\
\bottomrule
\end{tabular}
\caption{Kết quả benchmark tổng quan (cần bổ sung chi tiết test cases).}
\end{table}

\section{Phân tích trade-off}
Plan-RAG tăng chất lượng (accuracy/relevance) nhưng tăng latency do nhiều bước và nhiều LLM calls. 
Hướng tối ưu: caching, parallel execution, reranking, và tối ưu retrieval.

\begin{figure}[H]
\centering
\fbox{\parbox{0.9\textwidth}{
\textbf{Placeholder: biểu đồ đánh giá} \\
\textcolor{red}{TODO: Chèn biểu đồ cột/đường so sánh accuracy/relevance/latency.}
}}
\caption{Biểu đồ đánh giá (placeholder).}
\end{figure}

% ============================ Chapter 6 =======================
\chapter{Báo cáo lỗi quan trọng (12/01/2026) và kế hoạch khắc phục}

\section{Mô tả lỗi (User Experience)}
Khi user đã lập lịch trình và hỏi về \textbf{địa điểm/khách sạn đã chọn} (ví dụ "Bà Nà Hills có gì?"), hệ thống trả lời chung chung do truy xuất sai mục tiêu.

\section{Root Cause Analysis}
\begin{enumerate}
  \item Thiếu entity extraction (không trích tên spot/hotel từ query).
  \item Query MongoDB quá rộng theo \texttt{province\_id} thay vì theo tên/\_id cụ thể.
  \item Không tận dụng context \texttt{itinerary\_builder} chứa IDs chính xác.
\end{enumerate}

\section{Giải pháp đề xuất}
\subsection{Solution B – Quick Fix (Context-aware Query)}
Ưu tiên kiểm tra \texttt{context.itinerary\_builder} trước, nếu query khớp spot/hotel đã chọn thì truy vấn theo \_id.

\subsection{Solution A – LLM-based Entity Extraction (Recommended)}
Dùng LLM trích xuất entities (spots/hotels) từ query; kết hợp fuzzy matching (RapidFuzz) để xử lý typo.

\subsection{Solution C – Hybrid Approach (Best Practice)}
Decision tree: context fast-path $\rightarrow$ entity extraction $\rightarrow$ fuzzy match $\rightarrow$ broad search fallback.

\section{Kế hoạch kiểm thử (Testing \& Validation)}
\textcolor{red}{TODO: Bổ sung bảng test cases (TC-01..TC-07) và kết quả trước/sau fix (accuracy, relevance, latency).}

\section{Roadmap triển khai}
\begin{itemize}
  \item Phase 1: Quick Fix (P0) – 1 ngày.
  \item Phase 2: Entity Extraction (P1) – 2–3 ngày.
  \item Phase 3: Monitoring \& Optimization – ongoing.
\end{itemize}

% ============================ Chapter 7 =======================
\chapter{Kết luận và hướng phát triển}

\section{Kết luận}
Hệ thống Plan-RAG giúp tăng groundedness và chất lượng trả lời cho domain du lịch tiếng Việt, đồng thời mở rộng hỗ trợ multi-intent và tích hợp end-to-end với frontend. 
Báo cáo lỗi ngày 12/01/2026 chỉ ra một lỗ hổng quan trọng trong truy xuất thông tin chi tiết theo entity, cần bổ sung layer entity extraction + context-aware retrieval.

\section{Hướng phát triển}
\begin{itemize}
  \item Caching (intent/plan/response) theo TTL.
  \item Chạy song song experts (asyncio) và tối ưu DAG execution.
  \item Vector retrieval cho tất cả domain + reranking.
  \item Chuẩn hoá citation/sources trong response.
  \item Feedback loop học từ phản hồi người dùng.
\end{itemize}

% ============================ Appendices ======================
\appendix

\chapter{Phụ lục A: Định dạng JSON Intent và Execution Plan}
\section{ExtractedIntent}
\begin{lstlisting}[language=json,caption={Mẫu ExtractedIntent (rút gọn)}]
{
  "intent": "plan_trip",
  "mode": "traveler",
  "location": "Da Lat",
  "duration": 3,
  "budget": 5000000,
  "budget_level": "trung binh",
  "people_count": 2,
  "interests": ["photography", "food"],
  "keywords": ["chup anh", "dac san"],
  "confidence": 0.85
}
\end{lstlisting}

\section{ExecutionPlan (DAG SubTasks)}
\begin{lstlisting}[language=json,caption={Mẫu ExecutionPlan (rút gọn)}]
{
  "original_query": "Lich trinh 3 ngay Da Lat cho 2 nguoi tam 5 trieu",
  "intent": "plan_trip",
  "location": "Da Lat",
  "tasks": [
    {"task_id": "spots_1", "task_type": "find_spots", "priority": 1, "depends_on": []},
    {"task_id": "food_1", "task_type": "find_food", "priority": 1, "depends_on": []},
    {"task_id": "hotel_1", "task_type": "find_hotels","priority": 1, "depends_on": []},
    {"task_id": "itinerary_1", "task_type": "create_itinerary","priority": 2,
     "depends_on": ["spots_1","food_1","hotel_1"]},
    {"task_id": "cost_1", "task_type": "calculate_cost","priority": 3,
     "depends_on": ["itinerary_1"]}
  ],
  "execution_order": ["spots_1","food_1","hotel_1","itinerary_1","cost_1"]
}
\end{lstlisting}

\chapter{Phụ lục B: Placeholder cho Pipeline chi tiết}
\textcolor{red}{TODO: Chèn các pipeline chi tiết: (1) Retrieval pipeline (hybrid search), (2) Itinerary builder flow, (3) B2B analytics pipeline, (4) Monitoring/metrics pipeline.}

\chapter{Phụ lục C: Tài liệu tham khảo}
% Uncomment if you have references.bib
% \printbibliography

% If no bibliography, add manual references or remove this chapter
\textit{(Tài liệu tham khảo sẽ được bổ sung)}

\end{document}